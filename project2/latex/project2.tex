\documentclass[english,notitlepage]{revtex4-1}  % defines the basic parameters of the document
%For preview: skriv i terminal: latexmk -pdf -pvc filnavn



% if you want a single-column, remove reprint

% allows special characters (including æøå)
\usepackage[utf8]{inputenc}
\usepackage[english]{babel}


\usepackage{physics,amssymb}  % mathematical symbols (physics imports amsmath)
\include{amsmath}
\usepackage{graphicx}         % include graphics such as plots
\usepackage{xcolor}           % set colors
\usepackage{hyperref}         % automagic cross-referencing (this is GODLIKE)
\usepackage{subcaption}
\usepackage{listings}         % display code
\usepackage{float}
\usepackage{enumitem}
%\usepackage[section]{placeins}
\usepackage{algorithm}
\usepackage[noend]{algpseudocode}
\usepackage{tikz}
\usetikzlibrary{quantikz}

\hypersetup{ 
    colorlinks,
    linkcolor={red!50!black},
    citecolor={blue!50!black},
    urlcolor={blue!80!black}}



\begin{document}

\title{Project 2}      
\author{Alessio Canclini, Filip von der Lippe}          
\date{\today}                             
\noaffiliation                            % ignore this, but keep it.


\maketitle

\textit{Github repository: \url{https://github.com/Fslippe/FYS4150/tree/main/project2}}
\\
\\
Second order differential equation describing our buckling beam situation:
\begin{align}
    \gamma \frac{d^2u(x)}{dx^2} = - F u(x)
    \label{eq:diff}
\end{align}
Troughout this project we will be working with the scaled equation:
\begin{align}
    \frac{d^2u(\hat{x})}{d \hat{x}^2} = - \lambda u(\hat{x})
    \label{eq:scaled}
\end{align}
Where $\hat{x} \equiv x / L$ is a dimensionless variable, $\hat{x} \in [0,1]$ and $\lambda = \frac{FL^2}{\gamma}$.
\section*{Problem 1}
Using the defenintion $\hat{x} \equiv x / L$ to show that Eq. \ref*{eq:diff} can be written Eq. \ref*{eq:scaled}.
\begin{align*}
    \gamma \frac{d^2u(x)}{dx^2} &= - F u(x) \\
    \frac{d^2u(x)}{dx^2} &= - \frac{F}{\gamma} u(x)
    % ... fill
\end{align*}
\section*{Problem 2}

\section*{Problem 3}

\section*{Problem 4}

\section*{Problem 5}

\section*{Problem 6}


\end{document}