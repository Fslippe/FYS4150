
%% USEFUL LINKS:
%% -------------
%%
%% - UiO LaTeX guides:          https://www.mn.uio.no/ifi/tjenester/it/hjelp/latex/
%% - Mathematics:               https://en.wikibooks.org/wiki/LaTeX/Mathematics
%% - Physics:                   https://ctan.uib.no/macros/latex/contrib/physics/physics.pdf
%% - Basics of Tikz:            https://en.wikibooks.org/wiki/LaTeX/PGF/Tikz
%% - All the colors!            https://en.wikibooks.org/wiki/LaTeX/Colors
%% - How to make tables:        https://en.wikibooks.org/wiki/LaTeX/Tables
%% - Code listing styles:       https://en.wikibooks.org/wiki/LaTeX/Source_Code_Listings
%% - \includegraphics           https://en.wikibooks.org/wiki/LaTeX/Importing_Graphics
%% - Learn more about figures:  https://en.wikibooks.org/wiki/LaTeX/Floats,_Figures_and_Captions
%% - Automagic bibliography:    https://en.wikibooks.org/wiki/LaTeX/Bibliography_Management  (this one is kinda difficult the first time)
%%
%%                              (This document is of class "revtex4-1", the REVTeX Guide explains how the class works)
%%   REVTeX Guide:              http://www.physics.csbsju.edu/370/papers/Journal_Style_Manuals/auguide4-1.pdf
%%
%% COMPILING THE .pdf FILE IN THE LINUX IN THE TERMINAL
%% ----------------------------------------------------
%%
%% [terminal]$ pdflatex report_example.tex
%%
%% Run the command twice, always.
%%
%% When using references, footnotes, etc. you should run the following chain of commands:
%%
%% [terminal]$ pdflatex report_example.tex
%% [terminal]$ bibtex report_example
%% [terminal]$ pdflatex report_example.tex
%% [terminal]$ pdflatex report_example.tex
%%
%% This series of commands can of course be gathered into a single-line command:
%% [terminal]$ pdflatex report_example.tex && bibtex report_example.aux && pdflatex report_example.tex && pdflatex report_example.tex
%%
%% ----------------------------------------------------


\documentclass[english,notitlepage,reprint,nofootinbib]{revtex4-1}  % defines the basic parameters of the document
% For preview: skriv i terminal: latexmk -pdf -pvc filnavn
% If you want a single-column, remove "reprint"

% Allows special characters (including æøå)
\usepackage[utf8]{inputenc}
% \usepackage[english]{babel}

%% Note that you may need to download some of these packages manually, it depends on your setup.
%% I recommend downloading TeXMaker, because it includes a large library of the most common packages.

\usepackage{physics,amssymb}  % mathematical symbols (physics imports amsmath)
\include{amsmath}
\usepackage{graphicx}         % include graphics such as plots
\usepackage{xcolor}           % set colors
\usepackage{hyperref}         % automagic cross-referencing
\usepackage{listings}         % display code
\usepackage{subfigure}        % imports a lot of cool and useful figure commands
% \usepackage{float}
%\usepackage[section]{placeins}
\usepackage{algorithm}
\usepackage[noend]{algpseudocode}
\usepackage{subfigure}
\usepackage{tikz}
\usetikzlibrary{quantikz}
% defines the color of hyperref objects
% Blending two colors:  blue!80!black  =  80% blue and 20% black
\hypersetup{ % this is just my personal choice, feel free to change things
    colorlinks,
    linkcolor={red!50!black},
    citecolor={blue!50!black},
    urlcolor={blue!80!black}}


% ===========================================


\begin{document}

\title{Numerical simulation of a Penning trap}  % self-explanatory
\author{Alessio Canclini, Filip von der Lippe} % self-explanatory
\date{\today}                             % self-explanatory
\noaffiliation                            % ignore this, but keep it.

%This is how we create an abstract section.
\begin{abstract}
    NB! ABSTRACT HERE
    %We provide an overview of how to structure a scientific report. For concreteness, we consider the example of writing a report about an implementation of the midpoint rule of integration. For each section of the report we briefly discuss what the purpose of the given section is. We also provide examples of how to properly include equations, tables, algorithms, figures and references.
\end{abstract}
\maketitle


% ===========================================
\section{Introduction}
The purpose of this report is to present the study of the effects of a Penning trap through numerical simulations. The Penning trap is a device used to store or "trap" charged particles 
using static electric and magnetic fields as shown in figure \ref*{fig:Penning_trap}. These particles can then be used for a variety of experiments. Examples of this are the ALPHA, AEgIS and BASE 
experiments at CERN, these use Penning traps to control antimatter. 
The electric field is generated by two end caps (a), at the top and bottom, 
and a ring (b) (figure \ref*{fig:Penning_trap} only shows the ring cross-section).
This electric field restricts the particles' movement in the $z$ direction and the additional homogenous magnetic field 
hiders particles escaping in the $xy$-plane (radial direction) if it is strong enough. The magnetic field is set by 
a cylinder magnet (c) (figure \ref*{fig:Penning_trap} again only shows the ring cross-section). 

Materials to construct a physical Penning trap are very costly, we will therefore
be using a numerical approach to simulate a Penning trap. To implement such a simulation
we will be working with a system of coupled non-linear differential equations. These are very
difficult and often impossible to solve analytically. %An example some readers might be familiar with are the famous
%Navier-Stokes equations, the solving of which would be rewarded with a million dollar prize. 
In addition to the
material cost, the complexity of the equations therefore also motivates the use of numerical methods.

MAIN PURPOSE? 

Section \ref*{sec:methods} will describe the mathematical and physical background as well as concrete algorithms which in this case will be
implemented in C++, but can be written in any programming language.

In section \ref*{sec:results} we present a selection of results form different simulations and error analysis. This will include 
simulations with a single particle, 2 particles and 100 random particles. 

A detailed discussion of the algorithms' and results is presented in section \ref*{sec:discussion}, 
followed by a summary and potential for further experiments in section \ref*{sec:conclusion}.

\begin{figure}[H]
    \centering
    \includegraphics[width=.5\textwidth]{../figures/Penning_trap.pdf}
    \caption{This figure shows the idea of a Penning trap with a red positively charged particle in the center. 
    Here blue lines represent the electric field generated by a quadrupole consisting of end caps (a) and a ring electrode (b). 
    The red lines represent the magnetic field created by a surrounding cylinder magnet (c). 
    Illustration by Arian Kriesch taken from Wikimedia Commons.}
    \label{fig:Penning_trap}
\end{figure}

% ===========================================
\section{Methods}\label{sec:methods}
The physical laws used to implement the Penning trap simulation will be from electrodynamics and classical mechanics, we will not take quantum aspects into account.
The following equations will be used:

\begin{equation}\label{eq:el_field}
    \textbf{E} = - \nabla V
\end{equation}
$\textbf{E}$ is the electric field and $V$ the electric potential.

\begin{equation}\label{eq:el_at_r}
    \textbf{E(r)} = k_e \sum_{j=1}^{n} q_j \frac{\textbf{r} - \textbf{r}_j}{|\textbf{r} - \textbf{r}_j|^3}
\end{equation}
$\textbf{E(r)}$ is the electric field at a point \textbf{r}. This is set up by point charges ${q_1,...,q_n}$ at points ${\textbf{r}_1,...,\textbf{r}_n}$. This comes from \textbf{Coulomb's law}, stating the magnitude of force between to point charges. $k_e \approx 8.988 \cdot 10^9 N m^2 C^{-2}$ is the Coulomb constant.  

\begin{equation}\label{eq:lorentz}
    \textbf{F} = q\textbf{E} + q \textbf{v} \cross \textbf{B}
\end{equation}
This is the $\textbf{Lonretz force}$, the force $\textbf{F}$ on a particle 
with charge $q$, an electric field \textbf{E}, magnetic field \textbf{B} and velocity of the particle \textbf{v}.

\begin{equation}\label{eq:N2L}
    m \ddot{\textbf{r}} = \sum_i \textbf{F}_i
\end{equation}
Eq. \ref*{eq:N2L} is Newton's second law. Here $m$ is the mass of the particle and $\ddot{\textbf{r}} \equiv \frac{d^2 \textbf{r}}{dt^2}$ (the acceleration). 
Famously expressing that the sum of forces equals mass times acceleration.

\begin{equation}\label{eq:el_potential}
    V(x,y,z) = \frac{V_0}{2d^2} (2z^2 - x^2 - y^2)
\end{equation}
For this experiment we will be considering an ideal Penning trap for which the electric field $\textbf{E}$ is given by the electric potential $V$. 
Here $V_0$ is the potential applied to the electrodes. The trap will be approximated as a sphere where $d = \sqrt{z_0^2 + r_0^2 / 2}$ is the 
$\textit{charachteristic dimension}$ representing the length scale (or radius of the sphere)
for the region between electrodes. Here $z_0$ is the distance from the center to the end caps (a) and $r_0$ is the distance from the center 
to the surrounding ring (b).

\begin{equation}\label{eq:mag_field}
    \textbf{B} = B_0 \hat{e}_z = (0,0,B_0)
\end{equation}
$\textbf{B}$ is the homogenous magnetic field and is dictated by the field strength $B_0$. With $B_0 > 0$. 

\subsection*{Equations for single particle motion}
Now starting from Newton's second law and using the equations above we can express the time evolution of a single particle's motion.
The sum of forces will be the Lonretz force. Putting eq. \ref*{eq:lorentz} into eq. \ref*{eq:N2L} leads to:
\begin{equation}\label{eq:N2L_lorentz}
    m \ddot{\textbf{r}} = q\textbf{E} + q \textbf{v} \cross \textbf{B}
\end{equation}
Here $\ddot{\textbf{r}} = (\ddot{x},\ddot{y},\ddot{z})$ and $\textbf{v} = (\dot{x},\dot{y},\dot{z})$. Putting eq. \ref*{eq:el_potential} into eq. \ref*{eq:el_field} gives us:
\begin{equation}\label{eq:el_calculated}
    \textbf{E} = \left( x \frac{v_0}{d^2}, y \frac{v_0}{d^2}, -2z \frac{v_0}{d^2} \right)
\end{equation}
Now looking at $q \textbf{v} \cross \textbf{B}$ we have:
\begin{equation}\label{eq:cross_calculated}
    (q \dot{x},q \dot{y},q \dot{z}) \cross (0, 0, B_0) = (B_0 q \dot{y}, -B_0 q \dot{x}, 0)
\end{equation}
Finally substituting for $\textbf{E}$ and $q \textbf{v} \cross \textbf{B}$ in eq. \ref*{eq:N2L_lorentz} results in:
\begin{equation*}
    m \begin{pmatrix}
        \ddot{x} \\ \\ \ddot{y} \\ \\ \ddot{z}
    \end{pmatrix}
    = \begin{pmatrix}
        q x \frac{v_0}{d^2} \\ \\ q y \frac{v_0}{d^2} \\ \\ -2 q z \frac{v_0}{d^2}
    \end{pmatrix}
    + \begin{pmatrix}
        B_0 q \dot{y} \\ \\ -B_0 q \dot{x} \\ \\ 0
    \end{pmatrix}
\end{equation*}
Rewriting this as a set of equations leaves us with:
\begin{align}
    \ddot{x} - w_0 \dot{y} - \frac{1}{2}& w_z^2 x = 0 \label{eq:a_x}\\
    \ddot{y} + w_0 \dot{x} - \frac{1}{2}& w_z^2 y = 0 \label{eq:a_y}\\
    \ddot{z} + w_z^2& z = 0 \label{eq:a_z}
\end{align}
Where $w_0 = \frac{qB_0}{m}$ and $w_z^2 = \frac{2qV_0}{md^2}$. Taking a closer look at eq. \ref*{eq:a_z} we see that the general solution is:
\begin{equation}
    z = A \cos(w_z^2 t) + B \sin(w_z^2 t)
\end{equation}
eq. \ref*{eq:a_x} and \ref*{eq:a_y} are coupled, thus introducing a challenge. This can be resolved by introducing a complex function $f(t) = x(t) + iy(t)$ and rewriting them as a single differential equation. By introducing the complex function we have:
\begin{align}
    f(t) = x(t) + iy(t) \\
    \dot{f}(t) = \dot{x}(t) + i\dot{y}(t) \\
    \ddot{f}(t) = \ddot{x}(t) + i\ddot{y}(t)
\end{align}
Now multiplying eq. \ref*{eq:a_y} by $i$ gives:
\begin{equation}
    i\ddot{y} + iw_0 \dot{x} - i\frac{1}{2} w_z^2 y = 0 \label{eq:a_iy}
\end{equation}
eq. \ref*{eq:a_x} and \ref*{eq:a_iy} can then be summed:
\begin{equation}
    \ddot{x}+ i\ddot{y} - w_0 \dot{y} + iw_0 \dot{x} - \frac{1}{2} w_z^2 x - i\frac{1}{2} w_z^2 y= 0
\end{equation}
Finally, substituting for $f(t)$, $\dot{f}(t)$ and $\ddot{f}(t)$ shows that eq. \ref*{eq:a_x} and \ref*{eq:a_y} can be rewritten as a single differential equation for $f$:
\begin{equation}
    \ddot{f} + i w_0 \dot{f} - \frac{1}{2} w_z^2 f = 0 \label{eq:single_diff}
\end{equation}

The general solution to eq. \ref*{eq:single_diff} is:
\begin{equation}
    f(t) = A_+ e^{-i(w_+ t + \phi_+)} + A_- e^{-i(w_- t + \phi_-)} \label{eq:general_sol_f}
\end{equation}
where the amplitudes $A_+$ and $A_-$ are positive, $\phi_+$ and $\phi_-$ are constant phases, and
\begin{equation}
    w_\pm = \frac{w_0 \pm \sqrt{w_0^2 - 2 w_z^2}}{2}
\end{equation}
To obtain a bounded solution for the radial movement ($xy$-plane) of the particle we need to introduce some constraints on $w_0$ and $w_z$. In other words we will introduce some constraints that will ensure that $|f(t)| < \infty$ as $t \rightarrow \infty$.

Studying eq. \ref*{eq:general_sol_f} one notices that $|f(t)| \rightarrow \infty$ only if $w_\pm$ is complex. To avoid this, we introduce the limitation $w_0^2 - 2 w_z^2 \geq 0$, avoiding any negative values inside the square root and consequently limiting the result to real numbers. Rearranging this and remembering that $w_0 = \frac{qB_0}{m}$ and $w_z^2 = \frac{2qV_0}{md^2}$ leaves us with:
\begin{equation}
    \frac{q}{m} \geq \frac{4 V_0}{B_0^2 d^2}
\end{equation}
A constraint, that if satisfied, keeps the particle within the Penning trap.

Now to express the upper and lower bounds of the particles' distance from the origin in the $xy$-plane 
we start with eq. \ref*{eq:general_sol_f}. Through Taylor expansion of $e^{ix}$ we arrive at Euler's formula:
\begin{equation*}
    e^{ix} = \cos(x) + i \sin(x)
\end{equation*} 
Introducing $u =  w_+ t + \phi_+$ and $v = w_- t + \phi_-$, and using 
Euler's formula, eq. \ref*{eq:general_sol_f} can be rewritten as:
\begin{equation*}
    f(t) = A_+\left(\cos(u) - i \sin(u) \right) + A_-\left(\cos(v) - i \sin(v) \right)
\end{equation*}
The physical coordinates can then be found as $x(t) = \Re f(t)$ and  $y(t) = \Im f(t)$. Giving us:
\begin{align*}
    x(t) &= A_+ \cos(u) + A_- \cos(v) \\
    y(t) &= -A_+ \sin(u) - A_- \sin(v)
\end{align*}
Since our $xy$-plane is a circle, the particle's distance from the origin can be expressed as the radius $R = \sqrt{x^2 + y^2}$. 
Substituting for $x$ and $y$ we have:
\begin{align*}
    R = &[ (A_+ \cos(u) + A_- \cos(v))^2 \\ &+ (-A_+ \sin(u) - A_- \sin(v))^2 ]^{1/2}
\end{align*}
After expanding and rearranging we have:
\begin{align*}
   R = &[A_+^2(\cos^2(u) + \sin^2(u)) + A_-^2(\cos^2(v) + \sin^2(v)) \\ &+ 2 A_+ A_- (cos(u)\cos(v) + \sin(u)\sin(v))]^{1/2}
\end{align*}
Recognizing that $cos^2(u) + \sin^2(u) = 1$ and $cos(u)\cos(v) + \sin(u)\sin(v) = \cos(u-v)$ this can be simplified to:
\begin{align*}
    R = \sqrt{A_+^2 + A_-^2 + 2 A_+ A_- \cos(u-v)}
\end{align*}
We know that $\cos$ is a function with an upper bound of $1$ and lower bound of $-1$. 
The possible bounds for $R$ are consequently:
\begin{align*}
    R_+ &= \sqrt{A_+^2 +2A_+ A_- + A_-^2} = A_+ + A_- \\
    R_- &= \sqrt{A_+^2 -2A_+ A_- + A_-^2} = |A_+ - A_-|
\end{align*}

\subsection*{Equations for multiple particle motion}
For all simulations with more than one particle (when particle interaction is taken into account) 
the equations of motion will be coupled because of the Coulomb force between two point charges. 
Our numerical algorithms will therefore be solving the following equations:
\begin{align}
    \ddot{x}_i - w_{0,i} \dot{y}_i - \frac{1}{2}& w_{z,i}^2 x_i - k_e \frac{q_i}{m_i} \sum_{j\neq i}^{n} q_j \frac{x_i - x_j}{|\textbf{r}_i - \textbf{r}_j|^3} = 0 \label{eq:a_x_coupled}\\
    \ddot{y}_i + w_{0,i} \dot{x}_i - \frac{1}{2}& w_{z,i}^2 y_i - k_e \frac{q_i}{m_i} \sum_{j\neq i}^{n} q_j \frac{y_i - y_j}{|\textbf{r}_i - \textbf{r}_j|^3} = 0 \label{eq:a_y_coupled}\\
    \ddot{z}_i + w_{z,i}^2& z_i - k_e \frac{q_i}{m_i} \sum_{j\neq i}^{n} q_j \frac{z_i - z_j}{|\textbf{r}_i - \textbf{r}_j|^3} = 0 \label{eq:a_z_coupled}
\end{align}
Here the summation term is the Coulomb force and $i$ and $j$ are particle indices for $n$ particles 
with charges $\{q_1,...,q_n\}$ and masses $\{m_1,..,m_n\}$. 
% ===========================================
\subsection*{The algorithms}
To evolve our Penning trap system in time the following algorithms will be implemented to solve our particles' equations of motion. 
Consider the forward difference approximation for the first derivative based on Taylor expansion
\begin{equation}
    y_n' = \frac{y_{n+1}  - y_n}{h}, \quad h = t_{t-1} -t_n
\end{equation}
Rearranging this results in the forward Euler method which has a Global error of $\mathcal{O}(h)$, 
making it only first order accurate. The method is easy to implement and will be used for comparisons 
and checking of the main algorithm. 
%
% The algorithm for the midpoint rule is summarized in algorithm~\ref{algo:midpointrule}. The basic idea behind the algorithm is to divide the integration range into to $n$ small subintervals of length $h$, and on each such subinterval approximate the function $f(x)$ by a constant function. The value for this constant function is taken to be the value of $f(x)$ evaluated at the midpoint of the given subinterval --- hence the name of the method.
% %
\begin{figure}[H]
% NOTE: We only need \begin{figure} ... \end{figure} here because of a compatability issue between the 'revtex4-1' document class and the 'algorithm' environment.
    \begin{algorithm}[H]
    \caption{Forward Euler method}
    \label{algo:EUL}
        \begin{algorithmic}
            \Procedure{Forward Euler}{$y_0, h$}
            \State $y' = f(t_i,y_i)$        \Comment{Single fist-order diff. eq.}
            \State $n = 1 / h$ \Comment{Compute number of steps}
            \\
            \For{$i = 0, 1, 2, \ldots, n$}
            \State $y_{i+1} = y_i + h f(t_i, y_i)$  \Comment{Value at next time step} 
            \EndFor
            \EndProcedure
        \end{algorithmic}
    \end{algorithm}
\end{figure}

    Similarly to the forward Euler, Runge-Kutta fourth order (RK4) is based on Taylor expansion, but by adding several intermediate 
    steps to the computation of $y_{i+1}$ it generally yields better solutions for ODEs. RK4 has a global error of $\mathcal{O}(h^4)$.

    \begin{figure}[H]
        % NOTE: We only need \begin{figure} ... \end{figure} here because of a compatability issue between the 'revtex4-1' document class and the 'algorithm' environment.
            \begin{algorithm}[H]
            \caption{Runge-Kutta fourth order method}
            \label{algo:RK4}
                \begin{algorithmic}
                    \Procedure{Runge-Kutta fourth order}{$y_0, h$}
                    \State $y' = f(t_i,y_i)$        \Comment{Single fist-order diff. eq.}
                    \State $n = 1 / h$ \Comment{Compute number of steps}
                    \\
                    \For{$i = 0, 1, 2, \ldots, n$}
                    \State $k_1 = hf(t_i,y_i)$  \Comment{Intermediate step 1}
                    \State $k_2 = hf(t_i + h/2, y_i +k_1/2)$  \Comment{Intermediate step 2}
                    \State $k_3 = hf(t_i + h/2, y_i + k_2/2)$ \Comment{Intermediate step 3}
                    \State $k_4 = hf(t_i + h, y_i + k_3)$ \Comment{Intermediate step 4}
                    \State $y_{i+1} = y_i + \frac{1}{6}(k_1 + 2k_2 + 2k_3 + k_4)$ \Comment{Final algorithm}
                    \EndFor
                    \EndProcedure
                \end{algorithmic}
            \end{algorithm}
        \end{figure}

        The RK4 scheme will be our main method as this is a "gold standard" in numerical analysis. 
    Early tests with one single particle will be confirmed by an analytical solution, but for comparison of more complex 
    scenarios the simple forward Euler scheme will also be implemented.

    The Penning trap with time evolving algorithms will be implemented in an object-oriented C++ code, the plots are generated using Python.

    \subsection*{Particles}
    Our charged particles will be singly-charged Calcium ions $(Ca^+)$ with a mass of $40.077 u$.
    \begin{itemize}
        \item Particle 1:
            \subitem $(x_0,y_0,z_0) = (20, 0, 20) \mu m$
            \subitem $(v_{x,0}, v_{y,0}, v_{z,0}) = (0, 25, 0) \mu m / \mu s$  
        \item Particle 2:
            \subitem $(x_0,y_0,z_0) = (25, 25, 0) \mu m$
            \subitem $(v_{x,0}, v_{y,0}, v_{z,0}) = (0, 40, 5) \mu m / \mu s$  
    \end{itemize}
    Simulations with one particle will use Particle 1 whilst simulations with two particles will use both Particle 1 and Particle 2.
    For further simulations with 100 particles, Armadillo's \texttt{vec::randu()} function will be used to generate random initial 
    positions and velocities for each particle.

    \subsection*{Numbers and units}
% \textit{As demonstrated in algorithm~\ref{algo:midpointrule}, it is conventional to present algorithms in a way that is independent of any specific programming language. This ensures that it is the logic behind the algorithm that remains in focus, rather than the syntax of a particular programming language. In algorithm~\ref{algo:midpointrule} we have also demonstrated a common notation: The right-to-left arrow ($\leftarrow$) means that we assign the value of everything on the right to the variable on the left. This is nothing but how the ``='' symbol functions in most programming languages, but the arrow notation makes it clear that we are in fact assigning a value, rather than stating that two things are equal.}


% % ===========================================
\section{Results}\label{sec:results}
First we test our Penning trap's behavior for a single particle to see if it produces the expected motion in this simple case. 
All three solutions have been plotted in figure \ref*{fig:compare_analytic_tz} to examine weather the numerical and analytic results agree.
%single particle
\begin{figure}[H]
    \centering
    \includegraphics[width=.5\textwidth]{../figures/compare_analytic_tz.pdf}
    \caption{Here we see the vertical path of a single particle over 50 $\mu s$. The analytic, Euler and RK4 solutions are shown.}
    \label{fig:compare_analytic_tz}
\end{figure}
Figure \ref*{fig:compare_analytic_tz} shows that both the forward Euler and RK4 method show the same 
$z$ motion as the analytic solution for a single particle. The particle oscillates vertically.  

%two particles
\begin{figure}[H]
    \centering
    \includegraphics[width=.5\textwidth]{../figures/RK4_2_xy_no_interaction_xy.pdf}
    \caption{The horizontal ($xy$-plane) path of two particles without interaction.}
    \label{fig:RK4_2_xy_no_interaction_xy}
\end{figure}
In figure \ref*{fig:RK4_2_xy_no_interaction_xy} we see the motion of Particle 1 and Particle 2 in the $xy$-plane when they do not interact. 
That is, disregarding the Coulomb force. 
The paths have a very similar spiraling shape both resulting circular patterns around the origin. 

\begin{figure}[H]
    \centering
    \includegraphics[width=.5\textwidth]{../figures/RK4_2_xy_with_interaction_xy.pdf}
    \caption{The horizontal ($xy$-plane) path of two particles with interaction.}
    \label{fig:RK4_2_xy_with_interaction_xy}
\end{figure}
Figure \ref*{fig:RK4_2_xy_with_interaction_xy} shows the same two particles as figure \ref*{fig:RK4_2_xy_no_interaction_xy} with the exact same initial conditions, 
but with interaction between particles. We notice that the paths, although still spiraling, are not as circular.

% Phase spaces xy
\begin{figure}[H]
    \centering
    \includegraphics[width=.5\textwidth]{../figures/RK4_2_x_phasespace_no_interaction_phasespace_xy.pdf}
    \caption{The phase space $(x, v_x)$ of two particles without interaction.}
    \label{fig:RK4_2_x_phasespace_no_interaction_phasespace_xy}
\end{figure}

\begin{figure}[H]
    \centering
    \includegraphics[width=.5\textwidth]{../figures/RK4_2_x_phasespace_with_interaction_phasespace_xy.pdf}
    \caption{The phase space $(x, v_x)$ of two particles with interaction.}
    \label{fig:RK4_2_x_phasespace_with_interaction_phasespace_xy}
\end{figure}
The two figures above show the trajectories of Particle 1 and Particle 2 in the $(x,v_x)$ plane. 
Figure \ref*{fig:RK4_2_x_phasespace_no_interaction_phasespace_xy} is a simulation disregarding particle interaction whilst 
figure \ref*{fig:RK4_2_x_phasespace_with_interaction_phasespace_xy} shows the trajectories with particle interaction. We 
notice that the trajectories in figure \ref*{fig:RK4_2_x_phasespace_no_interaction_phasespace_xy} are fairly repetitive whilst 
figure \ref*{fig:RK4_2_x_phasespace_with_interaction_phasespace_xy} shows more variations and different final points to figure 
\ref*{fig:RK4_2_x_phasespace_no_interaction_phasespace_xy}.

% Phase spaces z
\begin{figure}[H]
    \centering
    \includegraphics[width=.5\textwidth]{../figures/RK4_2_z_phasespace_no_interaction_phasespace_xy.pdf}
    \caption{The phase space $(z, v_z)$ of two particles without interaction.}
    \label{fig:K4_2_z_phasespace_no_interaction_phasespace_xy}
\end{figure}

\begin{figure}[H]
    \centering
    \includegraphics[width=.5\textwidth]{../figures/RK4_2_z_phasespace_with_interaction_phasespace_xy.pdf}
    \caption{The phase space $(z, v_z)$ of two particles with interaction.}
    \label{fig:K4_2_z_phasespace_with_interaction_phasespace_xy}
\end{figure}
Figures \ref*{fig:K4_2_z_phasespace_no_interaction_phasespace_xy} and \ref*{fig:K4_2_z_phasespace_with_interaction_phasespace_xy} 
show the trajectories of Particle 1 and Particle 2 in the $(z,v_z)$ plane. In figure \ref*{fig:K4_2_z_phasespace_no_interaction_phasespace_xy}, 
without particle interactions, we notice what seems to be repetitive trajectories similarly to figure 
\ref*{fig:RK4_2_x_phasespace_no_interaction_phasespace_xy}. Figure \ref*{fig:K4_2_z_phasespace_with_interaction_phasespace_xy} 
which includes particle interactions on the other hand displays more irregular trajectories similarly to figure \ref*{fig:RK4_2_x_phasespace_with_interaction_phasespace_xy}.

%3D path plots
\begin{figure}[H]
    \centering
    \includegraphics[width=.5\textwidth]{../figures/3D_2_particles_no_interaction.pdf}
    \caption{Shows a 3D representation of two particle paths without interaction.}
    \label{fig:3D_2_particles_no_interaction}
\end{figure}

\begin{figure}[H]
    \centering
    \includegraphics[width=.5\textwidth]{../figures/3D_2_particles_with_interaction.pdf}
    \caption{Shows a 3D representation of two particle paths with interaction.}
    \label{fig:3D_2_particles_with_interaction}
\end{figure}
Looking at the 3D-plots in figures \ref*{fig:3D_2_particles_no_interaction} and \ref*{fig:3D_2_particles_with_interaction} 
it is difficult to follow the exact motion of the particles. Comparing figure \ref*{fig:3D_2_particles_no_interaction}, without 
particle interactions, to figure \ref*{fig:3D_2_particles_with_interaction}, with interactions, one notices that the start positions 
are identical, yet the finishing positions differ. Taking a close look it is also possible to see differences in the orange and blue 
lines in the two figures. 

%Relative error
\begin{figure}[H]
    \centering
    \includegraphics[width=.5\textwidth]{../figures/relative_error_Euler_norm.pdf}
    \caption{The plot shows the relative error for a single particle's motion over 50 $\mu s$ using the forward Euler approximation. 
            The four different lines show results for different step sizes $h = 50/N \mu s$ }
    \label{fig:relative_error_Euler_norm}
\end{figure}

\begin{figure}[H]
    \centering
    \includegraphics[width=.5\textwidth]{../figures/relative_error_RK4_norm.pdf}
    \caption{The plot shows the relative error for a single particle's motion over 50 $\mu s$ using the fourth order Runge-Kutta approximation.
            The four different lines show results for different step sizes $h = 50/N \mu s$ }
    \label{fig:relative_error_RK4_norm}
\end{figure}
Considering the case of a single particle to allow for a comparison to the analytic solution, figures \ref*{fig:relative_error_Euler_norm} 
and \ref*{fig:relative_error_RK4_norm} display the relative error of the forward Euler method (figure \ref*{fig:relative_error_Euler_norm}) 
and the RK4 method (figure \ref*{fig:relative_error_RK4_norm}) for different step sizes $h = 50/N \mu s$. Noticeably for both 
methods, the error decreases as the step size does too (increasing the number of points $N$). Comparing the two, one clearly 
sees that the RK4 method is more precise than the forward Euler.  

% ===========================================
\section{Discussion}\label{sec:discussion}
%
\textit{Note that you are free to merge the presentation and discussion of the results into a single section of your report. This can in many cases lead to a more fluid presentation. If you do this, we recommend you use ``Results and discussion'' or similar for the section title.}

From table \ref{tab:midpointruletab}, we note that our implementation reproduces the analytical results to four digits precision when the integration range is divided into $n = 10^4$ subintervals. This indicates that that our implementation of the algorithm is correct.

From figure \ref{fig:rel_err}, we see that $\log_{10}(\epsilon)$ decreases linearly with $\log_{2}(n)$. From this, it should be possible to extract the convergence rate of our implementation of the midpoint rule. From a theoretical point of view we know that the midpoint rule should have a convergence rate of $\mathcal{O}(h^2)$. To properly verify our implementation, we should have estimated the convergence rate from our results and compared it to this theoretical rate. Without doing so, we cannot know that the our implementation of the algorithm is correct, even though we have seen that the numerical approximation converges to the correct answer in \ref{tab:midpointruletab}.

\textit{Although this is a somewhat silly example, please note the following: We are to-the-point in our discussion of the results, and we only make strong claims about what we are actually certain about. In the discussion it is important to try to be as concise as possible --- long paragraphs that only make very general points are typically of limited interest. Note that we also highlight aspects of our analysis that could have been improved and that might form a topic for future work.}


% ===========================================
\section{Conclusion}\label{sec:conclusion}
\textit{In this section we state three things in a concise manner: what we have done, what we have found, and what should or could be done in the future.}

We have investigated an implementation of the midpoint rule for numerical integration. As a first validation test we have checked that our implementation of the method reproduces the analytical result for the definite integral of $f(x) = x^3$ on $x \in [0,1]$, achieving a four-digit precision when the integration range is divided into $n=10^4$ subintervals. Furthermore, we have presented results for how the relative error of the method varies with the number of subintervals. To use these results to extract a precise estimate for the convergence rate of the method remains a topic for future work. As such, while our implementation of the midpoint rule has passed the initial validation tests, more work is needed to fully assess the validity of the implementation.

\onecolumngrid

%\bibliographystyle{apalike}
\bibliography{ref}


\end{document}
