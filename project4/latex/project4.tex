\documentclass[english,notitlepage,reprint,nofootinbib]{revtex4-1}  % defines the basic parameters of the document
% For preview: skriv i terminal: latexmk -pdf -pvc filnavn
% If you want a single-column, remove "reprint"

% Allows special characters (including æøå)
\usepackage[utf8]{inputenc}
% \usepackage[english]{babel}

%% Note that you may need to download some of these packages manually, it depends on your setup.
%% I recommend downloading TeXMaker, because it includes a large library of the most common packages.

\usepackage{physics,amssymb}  % mathematical symbols (physics imports amsmath)
\include{amsmath}
\usepackage{graphicx}         % include graphics such as plots
\usepackage{xcolor}           % set colors
\usepackage{hyperref}         % automagic cross-referencing
\usepackage{listings}         % display code
\usepackage{subfigure}        % imports a lot of cool and useful figure commands
% \usepackage{float}
%\usepackage[section]{placeins}
\usepackage{algorithm}
\usepackage[noend]{algpseudocode}
\usepackage{subfigure}
\usepackage{tikz}
\usetikzlibrary{quantikz}
% defines the color of hyperref objects
% Blending two colors:  blue!80!black  =  80% blue and 20% black
\hypersetup{ % this is just my personal choice, feel free to change things
    colorlinks,
    linkcolor={red!50!black},
    citecolor={blue!50!black},
    urlcolor={blue!80!black}}


% ===========================================


\begin{document}

\title{Markov Chain Monte Carlo approximation \\of the infinite lattice Ising model}  % self-explanatory
\author{Alessio Canclini, Filip von der Lippe} % self-explanatory
\date{\today}                             % self-explanatory
\noaffiliation                            % ignore this, but keep it.

%This is how we create an abstract section.
\begin{abstract}
    NB! Abstract here
\end{abstract}
\maketitle


% ===========================================
\section{Introduction}
This report will explore temperature-dependent behavior in ferromagnetism using the two-dimensional \textbf{Ising model}. The main purpose is determining a numerical estimation of the \textbf{critical temperature} at which the system transitions from a magnetized to a non-magnetized phase. 

... more about Ising model and phase transitions
+ Mote Carlo methods and Markov chains.

% ===========================================
\section{Methods}\label{sec:methods}
The square 2D lattices for our Ising model will have a length of $L$ containing $N$ spins with the relation $N = L^2$. Each spin $s_i$ will have two possible states of 
\begin{equation}
    s_i = -1 \text{ or } s_i = +1.
\end{equation}
The total spin state or \textbf{spin configuration} of a lattice will be represented as $\textbf{s} = (s_1, s_2, ..., s_N)$. The following equations will be used:
\begin{equation}
    E(\textbf{s}) = - J \sum^N_{\langle kl \rangle} s_k s_l
\end{equation}
$E$ is the total energy where $\langle kl \rangle$ denotes the sum going over all \textit{neighboring pairs} of spins avoiding double-counting. $J$ is the \textbf{coupling constant} simply setting the energy associated with spin interactions. \textit{Periodic boundary conditions} will be implemented allowing all spins to have four neighbors.
\begin{equation}
    M(\textbf{s}) = \sum^N_i s_i
\end{equation}
Here $M$ is the magnetization of the entire system expressed as a sum over all spins.
\begin{equation}
    \epsilon(\textbf{s}) = \frac{E(\textbf{s})}{N}
\end{equation}
\begin{equation}
    m(\textbf{s}) = \frac{M(\textbf{s})}{N}
\end{equation}
$\epsilon$ is the energy per spin and $m$ the magnetization per spin. These will be used to compare results.
\begin{equation}
    \beta = \frac{1}{k_B T}
\end{equation}
$\beta$ describes the ``inverse temperature'' with the systems' temperature $T$ and the Boltzmann constant $k_B$.
\begin{equation}
    Z = \sum_{\text{all possible \textbf{s}}} e^{- \beta E(\textbf{s})}
\end{equation}
$Z$ is the partition function. This, the `inverse temperature'' and the total energy of the system appear in the \textit{Boltzmann distridution} given bellow.
\begin{equation}
    p(\textbf{s};T) = \frac{1}{Z} e^{-\beta E(\textbf{s})}
\end{equation}

For comparison with early numerical implementations we will fist consider an analytical solution. The following table \ref*{tab:analytic} summarizes all sixteen possible \textbf{spin configurations} of a $2 \cross 2$ lattice with \textit{periodic boundary conditions}.

% Please add the following required packages to your document preamble:
% \usepackage[table,xcdraw]{xcolor}
% If you use beamer only pass "xcolor=table" option, i.e. \documentclass[xcolor=table]{beamer}
\begin{table}[H]
    \centering
    \caption{Analytic values for the sixteen \textbf{spin configurations} of $2 \cross 2$ Ising model lattice.} 
    \begin{tabular}{|l|l|l|l|}
    \hline
    \begin{tabular}[c]{@{}l@{}}Nr. of spins \\ in state +1\end{tabular} & \begin{tabular}[c]{@{}l@{}}Total \\ energy\end{tabular} & \begin{tabular}[c]{@{}l@{}}Total \\ magnetization\end{tabular} & Degeneracy \\ \hline
    \hline
    0   & -8 & -4  & 1 \\    \hline
    1   & 0  & -2  & 4 \\    \hline
    1   & 0  & -2  & 4 \\    \hline
    1   & 0  & -2  & 4 \\    \hline
    1   & 0  & -2  & 4 \\    \hline
    2   & 0  & 0   & 4 \\    \hline
    2   & 0  & 0   & 4 \\    \hline
    2   & 8  & 0   & 2 \\    \hline
    2   & 0  & 0   & 4 \\    \hline
    2   & 0  & 0   & 4 \\    \hline
    2   & 8  & 0   & 2 \\    \hline
    3   & 0  & 2   & 4 \\    \hline
    3   & 0  & 2   & 4 \\    \hline
    3   & 0  & 2   & 4 \\    \hline
    3   & 0  & 2   & 4 \\    \hline
    4   & -8 & 4   & 1 \\    \hline                                 
    \end{tabular} \label{tab:analytic}
\end{table} 
Based on the values in table \ref{tab:analytic} we derive the specific analytical expressions bellow for the $2 \cross 2$ lattice case.
% ===========================================
\section{Results}\label{sec:results}



% ===========================================
\section{Discussion}\label{sec:discussion}
%


% ===========================================
\section{Conclusion}\label{sec:conclusion}



\onecolumngrid

%\bibliographystyle{apalike}
\bibliography{ref}


\end{document}