\documentclass[english,notitlepage,reprint,nofootinbib]{revtex4-1}  % defines the basic parameters of the document
% For preview: skriv i terminal: latexmk -pdf -pvc filnavn
% If you want a single-column, remove "reprint"

% Allows special characters (including æøå)
\usepackage[utf8]{inputenc}
% \usepackage[english]{babel}

%% Note that you may need to download some of these packages manually, it depends on your setup.
%% I recommend downloading TeXMaker, because it includes a large library of the most common packages.

\usepackage{physics,amssymb}  % mathematical symbols (physics imports amsmath)
\include{amsmath}
\usepackage{graphicx}         % include graphics such as plots
\usepackage{xcolor}           % set colors
\usepackage{hyperref}         % automagic cross-referencing
\usepackage{listings}         % display code
\usepackage{subfigure}        % imports a lot of cool and useful figure commands
% \usepackage{float}
%\usepackage[section]{placeins}
\usepackage{algorithm}
\usepackage[noend]{algpseudocode}
\usepackage{subfigure}
\usepackage{tikz}
\usetikzlibrary{quantikz}
% defines the color of hyperref objects
% Blending two colors:  blue!80!black  =  80% blue and 20% black
\hypersetup{ % this is just my personal choice, feel free to change things
    colorlinks,
    linkcolor={red!50!black},
    citecolor={blue!50!black},
    urlcolor={blue!80!black}}


% ===========================================


\begin{document}

\title{Markov Chain Monte Carlo approximation \\of the infinite lattice Ising model}  % self-explanatory
\author{Alessio Canclini, Filip von der Lippe} % self-explanatory
\date{\today}                             % self-explanatory
\noaffiliation                            % ignore this, but keep it.

%This is how we create an abstract section.
\begin{abstract}
    NB! abstract here
\end{abstract}
\maketitle


% ===========================================
\section{Introduction}


% ===========================================
\section{Methods}\label{sec:methods}
For comparison with early numerical implementations we will fist consider an analytical solution. The following table \ref*{tab:analytic} summarizes all sixteen possible \textbf{spin configurations} of a $2 \cross 2$ lattice with \textit{periodic boundary conditions}.

% Please add the following required packages to your document preamble:
% \usepackage[table,xcdraw]{xcolor}
% If you use beamer only pass "xcolor=table" option, i.e. \documentclass[xcolor=table]{beamer}
\begin{table}[H]
    \centering
    \caption{Analytical values for sixteen \textbf{spin configurations} of $2 \cross 2$ Ising model lattice.} 
    \begin{tabular}{|l|l|l|l|}
    \hline
    \begin{tabular}[c]{@{}l@{}}Nr of spins \\ in state +1\end{tabular} & \begin{tabular}[c]{@{}l@{}}Total \\ energy\end{tabular} & \begin{tabular}[c]{@{}l@{}}Total \\ magnetization\end{tabular} & Degeneracy \\ \hline
    \hline
    0   & -8 & -4   & 1 \\    \hline
    1   & 0  & -2   & 4 \\    \hline
    1   & 0  & -2   & 4 \\    \hline
    1   & 0  & -2   & 4 \\    \hline
    1   & 0  & -2   & 4 \\    \hline
    2   & 0  & 0    & 4 \\    \hline
    2   & 0  & 0    & 4 \\    \hline
    2   & 8  & 0    & 2 \\    \hline
    2   & 0  & 0    & 4 \\    \hline
    2   & 0  & 0    & 4 \\    \hline
    2   & 8  & 0    & 2 \\    \hline
    3   & 0  & 2    & 4 \\    \hline
    3   & 0  & 2    & 4 \\    \hline
    3   & 0  & 2    & 4 \\    \hline
    3   & 0  & 2    & 4 \\    \hline
    4   & -8 & 4    & 1 \\    \hline                                 
    \end{tabular} \label{tab:analytic}
    \end{table} 
% ===========================================
\section{Results}\label{sec:results}



% ===========================================
\section{Discussion}\label{sec:discussion}
%


% ===========================================
\section{Conclusion}\label{sec:conclusion}



\onecolumngrid

%\bibliographystyle{apalike}
\bibliography{ref}


\end{document}