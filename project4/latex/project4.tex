\documentclass[english,notitlepage,reprint,nofootinbib]{revtex4-1}  % defines the basic parameters of the document
% For preview: skriv i terminal: latexmk -pdf -pvc filnavn
% If you want a single-column, remove "reprint"

% Allows special characters (including æøå)
\usepackage[utf8]{inputenc}
% \usepackage[english]{babel}

%% Note that you may need to download some of these packages manually, it depends on your setup.
%% I recommend downloading TeXMaker, because it includes a large library of the most common packages.

\usepackage{physics,amssymb}  % mathematical symbols (physics imports amsmath)
\include{amsmath}
\usepackage{graphicx}         % include graphics such as plots
\usepackage{xcolor}           % set colors
\usepackage{hyperref}         % automagic cross-referencing
\usepackage{listings}         % display code
\usepackage{subfigure}        % imports a lot of cool and useful figure commands
% \usepackage{float}
%\usepackage[section]{placeins}
\usepackage{algorithm}
\usepackage[noend]{algpseudocode}
\usepackage{subfigure}
\usepackage{tikz}
\usepackage{mathrsfs}

\usetikzlibrary{quantikz}
% defines the color of hyperref objects
% Blending two colors:  blue!80!black  =  80% blue and 20% black
\hypersetup{ % this is just my personal choice, feel free to change things
    colorlinks,
    linkcolor={red!50!black},
    citecolor={blue!50!black},
    urlcolor={blue!80!black}}


% ===========================================


\begin{document}

\title{Markov Chain Monte Carlo approximation \\of the 2D Ising model}  % self-explanatory
\author{Alessio Canclini, Filip von der Lippe} % self-explanatory
\date{\today}                             % self-explanatory
\noaffiliation                            % ignore this, but keep it.

%This is how we create an abstract section.
\begin{abstract}
    NB! Abstract here
\end{abstract}
\maketitle


% ===========================================
\section{Introduction}
This report will explore temperature-dependent behavior in ferromagnetism using the two-dimensional \textbf{Ising model}. The main purpose is determining a numerical estimation of the \textbf{critical temperature} at which the system transitions from a magnetized to a non-magnetized phase. 

... more about Ising model and phase transitions
+ Mote Carlo methods and Markov chains. \cite{compendium}

% ===========================================
\section{Methods}\label{sec:methods}
The square 2D lattices for our Ising model will have a length of $L$ containing $N$ spins with the relation $N = L^2$. Each spin $s_i$ will have two possible states of 
\begin{equation*}
    s_i = -1 \text{ or } s_i = +1.
\end{equation*}
The total spin state or \textbf{spin configuration} of a lattice will be represented as $\textbf{s} = (s_1, s_2, ..., s_N)$. In its simplest form the total energy of the system is expressed as
\begin{equation*}
    E(\textbf{s}) = - J \sum^N_{\langle kl \rangle} s_k s_l - \mathscr{B} \sum^N_{k} s_k.
\end{equation*}
Here $\mathscr{B}$ is an external magnetic field. Since we will be looking at the Ising model without an external magnetic field the equation will be simplified to
\begin{equation}
    E(\textbf{s}) = - J \sum^N_{\langle kl \rangle} s_k s_l,
\end{equation}
where $\langle kl \rangle$ denotes the sum going over all \textit{neighboring pairs} of spins avoiding double-counting. $J$ is the \textbf{coupling constant} simply setting the energy associated with spin interactions. \textit{Periodic boundary conditions} will be implemented allowing all spins to have four neighbors.
\begin{equation}
    M(\textbf{s}) = \sum^N_i s_i
\end{equation}
is the magnetization of the entire system expressed as a sum over all spins. The energy per spin is
\begin{equation}
    \epsilon(\textbf{s}) = \frac{E(\textbf{s})}{N}
\end{equation}
and the magnetization per spin is given by
\begin{equation}
    m(\textbf{s}) = \frac{M(\textbf{s})}{N}.
\end{equation}
These will be used to compare and analyze results.
\begin{equation}
    \beta = \frac{1}{k_B T}
\end{equation}
describes the ``inverse temperature'' with the systems' temperature $T$ and the Boltzmann constant $k_B$.
\begin{equation}
    Z = \sum_{\text{all possible \textbf{s}}} e^{- \beta E(\textbf{s})}
\end{equation}
represents the partition function. This, the `inverse temperature'' and the total energy of the system appear in the \textit{Boltzmann distribution},
\begin{equation}
    p(\textbf{s};T) = \frac{1}{Z} e^{-\beta E(\textbf{s})}. \label{eq:prob_dist}
\end{equation}
This will be the probability distribution used for random sampling in our Monte Carlo approach. 


For comparison with early numerical implementations we will fist consider an analytical solution. The following table \ref*{tab:analytic} summarizes all sixteen possible \textbf{spin configurations} of a $2 \cross 2$ lattice with \textit{periodic boundary conditions}.

% Please add the following required packages to your document preamble:
% \usepackage[table,xcdraw]{xcolor}
% If you use beamer only pass "xcolor=table" option, i.e. \documentclass[xcolor=table]{beamer}
\begin{table}[H]
    \centering
    \caption{Analytic values for the sixteen \textbf{spin configurations} of $2 \cross 2$ Ising model lattice.} 
    \begin{tabular}{|l|l|l|l|}
    \hline
    \begin{tabular}[c]{@{}l@{}}Nr. of spins \\ in state +1\end{tabular} & Degeneracy & \begin{tabular}[c]{@{}l@{}}Total \\ energy\end{tabular} & \begin{tabular}[c]{@{}l@{}}Total \\ magnetization\end{tabular} \\ \hline
    \hline
    0    & 1 & -8J & -4  \\    \hline
    1    & 4 & 0  & -2  \\    \hline
    2    & 4 & 0  & 0   \\    \hline
    2    & 2 & 8J  & 0   \\    \hline
    3    & 4 & 0  & 2   \\    \hline
    4    & 1 & -8J & 4   \\    \hline                                 
    \end{tabular} \label{tab:analytic}
\end{table} 
Based on the values in table \ref{tab:analytic} we derive the specific analytical expressions for the $2 \cross 2$ lattice case. More comprehensive calculations of the analytic solutions can be found in appendix \ref{appendix:A}. The specific partition function becomes
\begin{equation}
    Z = 2 e^{\beta 8J} + 2 e^{- \beta 8J} + 12 = 4 (\cosh(8 \beta J) + 3)
\end{equation}
Additionally we will calculate a few expectation values for which the general formula is given as
\begin{equation}
    \langle A \rangle = \sum_s A_s p(s).
\end{equation}
This is a sum over all spin states $s_i$. Here $p(s)$ is a chosen probability distribution, in our this case the Boltzmann distribution in eq. \ref{eq:prob_dist}.
\begin{equation}
    \langle E \rangle = \frac{16J}{Z} \left( e^{-\beta 8J} - e^{\beta 8J} \right)
\end{equation}
\begin{equation}
    \langle E^2 \rangle = \frac{128ß J^2}{Z} \left( e^{-\beta 8J} + e^{\beta 8J} \right)
\end{equation}
\begin{equation}
    \langle \epsilon \rangle = \frac{4J}{Z} \left( e^{-\beta 8J} - e^{\beta 8J} \right)
\end{equation}
\begin{equation}
    \langle \epsilon^2 \rangle = \frac{32 J^2}{Z} \left( e^{-\beta 8J} + e^{\beta 8J} \right)
\end{equation}
\begin{equation}
    \langle |M| \rangle = \frac{8J}{Z} \left( e^{\beta 8J} + 2 \right)
\end{equation}
\begin{equation}
    \langle M^2 \rangle = \frac{8J}{Z} \left( e^{\beta 8J} + 1 \right)
\end{equation}
\begin{equation}
    \langle |m| \rangle = \frac{2}{Z} \left( e^{\beta 8J} + 2 \right)
\end{equation}
\begin{equation}
    \langle m^2 \rangle = \frac{2}{Z} \left( e^{\beta 8J} + 1 \right)
\end{equation}

NB! Missing analytic $C_V$ and $X$
\\ \\

The Monte Carlo method will repeatedly require the Boltzmann factor $e^{-\beta \Delta E}$. The energy shift induced by flipping a single spin 
\begin{equation}
    \Delta E = E_{\text{after}} - E_{\text{before}}
\end{equation}
can only take five possible values in a 2D-lattice of arbitrary size. This is shown in appendix \ref{appendix:B}. These values are
\begin{equation}
    \Delta E = 0, -4J, 4J, -8J, 8J.
\end{equation}
To avoid repeatedly calling the exponential function we pre-compute the corresponding Boltzmann factors in an array and use the correct one by implementing a set of if-tests.

% ===========================================
\section{Results}\label{sec:results}



% ===========================================
\section{Discussion}\label{sec:discussion}
%


% ===========================================
\section{Conclusion}\label{sec:conclusion}



\onecolumngrid

%\bibliographystyle{apalike}
\bibliography{ref}

\newpage
\appendix
\section{Analytical solutions for a 2$\cross$2 lattice}\label{appendix:A}

\section{Possible $\Delta E$ values}\label{appendix:B}

\begin{table}[H]
\centering
\begin{tabular}{llll}
   & $\uparrow$ &    \\
$\uparrow$ & $\uparrow$ & $\uparrow$ \\
   & $\uparrow$ &    \\
   &    &   
\end{tabular}
\end{table}

\end{document}