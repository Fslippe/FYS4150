\documentclass[english,notitlepage,reprint,nofootinbib]{revtex4-1}  % defines the basic parameters of the document
% For preview: skriv i terminal: latexmk -pdf -pvc filnavn
% If you want a single-column, remove "reprint"
% Allows special characters (including æøå)
\usepackage[utf8]{inputenc}
% \usepackage[english]{babel}

%% Note that you may need to download some of these packages manually, it depends on your setup.
%% I recommend downloading TeXMaker, because it includes a large library of the most common packages.

\usepackage{physics,amssymb}  % mathematical symbols (physics imports amsmath)
\include{amsmath}
\usepackage{graphicx}         % include graphics such as plots
\usepackage{xcolor}           % set colors
\usepackage{hyperref}         % automagic cross-referencing
\usepackage{listings}         % display code
%\usepackage{subfigure}        % imports a lot of cool and useful figure commands
\usepackage{float}
%\usepackage[section]{placeins}
\usepackage{algorithm}
\usepackage[noend]{algpseudocode}
\usepackage{subfigure}
\usepackage{tikz}
\usepackage{mathrsfs}
\bibliographystyle{apsrev4-1}
\usetikzlibrary{quantikz}
% defines the color of hyperref objects
% Blending two colors:  blue!80!black  =  80% blue and 20% black
\hypersetup{ % this is just my personal choice, feel free to change things
    colorlinks,
    linkcolor={red!50!black},
    citecolor={blue!50!black},
    urlcolor={blue!80!black}}
% ===========================================


\begin{document}

\title{A study of phase transition in the 2D Ising model\\using Markov Chain Monte Carlo simulation}  % self-explanatory
\author{Alessio Canclini, Filip von der Lippe} % self-explanatory
\date{\today}                             % self-explanatory
\noaffiliation                            % ignore this, but keep it.



\begin{table}
    \centering
    \caption{Time used for different number of threads in parall    elization at temperature level. A quad-core, 8 thread CPU has been used for the ising model initialized with a $5\times5$ lattice with randomized initial spins and temperature $T= 1$ $J/k_B$}
    \label{tab:timing}
    \begin{tabular}{|c|c|c|c|}
        \hline
        Number of threads & 1     & 4    & 8    \\
        \hline
        Time used (s)     & 29.12 & 8.42 & 7.32 \\
        \hline
        speed up factor   & 1     & 0.29 & 0.25 \\
        \hline
    \end{tabular}
\end{table}
In table \ref{tab:timing} we see that parallelizing our code can reduce computation time a factor of 0.25 using 8 threads for an 8 thread CPU. We also see that a 4 thread parallelization performs well with a 0.29 reduction.
% ===========================================


From our timing test using parallelization at temperature level we saw a great reduction in computation time when using more threads. Here it is important to notice that these timings are performed on a quad-core 8 thread CPU and can only be reproduced using a similar
\end{document}