\documentclass[english,notitlepage,reprint,nofootinbib]{revtex4-1}  % defines the basic parameters of the document
% For preview: skriv i terminal: latexmk -pdf -pvc filnavn
% If you want a single-column, remove "reprint"

% Allows special characters (including æøå)
\usepackage[utf8]{inputenc}
% \usepackage[english]{babel}

%% Note that you may need to download some of these packages manually, it depends on your setup.
%% I recommend downloading TeXMaker, because it includes a large library of the most common packages.

\usepackage{physics,amssymb}  % mathematical symbols (physics imports amsmath)
\include{amsmath}
\usepackage{graphicx}         % include graphics such as plots
\usepackage{xcolor}           % set colors
\usepackage{hyperref}         % automagic cross-referencing
\usepackage{listings}         % display code
\usepackage{subfigure}        % imports a lot of cool and useful figure commands
\usepackage{float}
%\usepackage[section]{placeins}
\usepackage{algorithm}
\usepackage[noend]{algpseudocode}
\usepackage{subfigure}
\usepackage{tikz}
\usetikzlibrary{quantikz}
% defines the color of hyperref objects
% Blending two colors:  blue!80!black  =  80% blue and 20% black
\hypersetup{ % this is just my personal choice, feel free to change things
    colorlinks,
    linkcolor={red!50!black},
    citecolor={blue!50!black},
    urlcolor={blue!80!black}}


% ===========================================


\begin{document}

\title{Numerical simulation of the 2+1 dimensional Schrödinger equation}  % self-explanatory
\author{Alessio Canclini, Filip von der Lippe} % self-explanatory
\date{\today}                             % self-explanatory
\noaffiliation                            % ignore this, but keep it.

%This is how we create an abstract section.
\begin{abstract}
   NB! Abstract here!
\end{abstract}
\maketitle


% ===========================================
\section{Introduction}
%
Looking at the world around you there are an incredible variety of processes occurring. This can be the weather, how heat diffuses through your cooking utensils and burns you, or how light behaves and allows you to see it all. Most such physical processes are amazingly complex and depend on many variables. Partial differential equations (PDEs) allow us to model such processes.
These PDEs can be incredibly difficult and often impossible to solve analytically. An example some readers might be familiar with are the famous Navier-Stokes equations, the solving of which would be rewarded with a million dollar prize. Thus, to simulate and at least approximate solutions to these equations we utilize the rapidly evolving tool of computational simulation by implementing a variety of finite difference schemes and letting a computer do the repetitive work. 
We will use the Crank-Nicholson scheme to solve a possibly even more famous PDE, the time-dependent Schrödinger equation. The equation is simplified to our specific case where we simulate arguably the most famous experiment in physics; the double slit experiment. It was originally performed by Thomas Young in 1802 resulting in the first demonstration of wave-particle duality. 

Crank-Nicholson

conservation of probability to check stability

% ===========================================
\section{Methods}\label{sec:methods}
%
To simulate the double-slit-in-a-box experiment we use the following theoretical framework. The time-dependent Schrödinger equation's general formulation is
\begin{equation}
    i \hbar \frac{d}{dt} | \Psi \rangle = \hat{H} | \Psi \rangle.
\end{equation}
Here $| \Psi \rangle$ is the quantum state and $\hat{H}$ is the Hamiltonian operator. For our purposes we consider a single, non-relativistic particle in two spatial dimensions. This allows $| \Psi \rangle$ to be expressed as $ \Psi (x,y,t)$, a complex-valued function. In this case the Schrödinger equation can be expressed as
\begin{align}
    i \hbar \frac{\partial}{\partial t} \Psi(x,y,t) = - \frac{\hbar^2}{2m} \left( \frac{\partial^2}{\partial x^2} + \frac{\partial}{\partial y^2}\right) \Psi(x,y,t) \\
    + V(x,y,t) \Psi (x,y,t).
\end{align}
In the first term on the RHS, $- \frac{\hbar^2}{2m} \frac{\partial^2 \Psi}{\partial x^2}$ and $- \frac{\hbar^2}{2m} \frac{\partial^2 \Psi}{\partial x^2}$ express kinetic energy equivalent to $\frac{p^2}{2m}$ in classical physics. Here $m$ is the particle mass. Only the case of a time-independent potential $V = V(x,y)$ is considered. Working in this kind of position space the Born rule is
\begin{align}
    p(x,y;t) = |\Psi(x,y,t)|^2 = \Psi^{\ast} (x,y,t) \Psi(x,y,t).
\end{align}
Here $p(x,y;t)$ is the probability density of a particle being detected at a position $(x,y)$ at a time $t$. Continuing we assume that all dimensions have been scaled away. This leaves us with a dimensionless Schrödinger equation
\begin{equation}
    i \frac{\partial u}{\partial t} = - \frac{\partial^2 u}{\partial y^2} - \frac{\partial^2 u}{\partial y^2} + v(x,y)u. \label{eq:wave_eq}
\end{equation}
$v(x,y)$ is some potential and $u = u(x,y,t)$ our ``wave function'' which will hold a complex value ($u \in \mathbb{C}$). With this new notation the Born rule becomes
\begin{equation}
    p(x,y,;t) = |u(x,y,t)|^2 = u^{\ast} (x,y,t) u(x,y,t).
\end{equation}
Here we assume that the wave function $u$ has been properly normalized.

\subsection*{Initial and boundary conditions}
Dirichlet boundary conditions are implemented in the $xy$-plane as follows.
\begin{itemize}
    \item $u(x=0,y,t) = 0$
    \item $u(x=1,y,t) = 0$
    \item $u(x,y=0,t) = 0$
    \item $u(x,y=1,t) = 0$
\end{itemize}

The initial wave function $u_{ij}^0$ is given by the unnormalized quantum mechanical Gaussian wave packet
\begin{align}
    u(x,y,t = 0) = e^{-\frac{(x-x_c)^2}{2 \sigma_x^2} -\frac{(y-y_c)^2}{2 \sigma_y^2}+ ip_x (x-x_c)+ ip_y (y-y_c)}.
\end{align}
This is then normalized such that
\begin{equation}
    \sum_{i,j} u_{ij}^{0*} u_{ij}^0 = 1,
\end{equation}
where $^*$ denotes the complex conjugate.
\subsection*{The Crank-Nicholson scheme}
Using the Crank-Nicholson scheme, eq. \ref{eq:wave_eq} is discretized as
\begin{align}
    &u_{ij}^{n+1} - r \left[ u_{i+1,j}^{n+1}- 2 u_{ij}^{n+1} + u_{i-1,j}^{n+1} \right] \\
    &- r \left[ u_{i,j+1}^{n+1}- 2 u_{ij}^{n+1} + u_{i,j-1}^{n+1} \right]
    + \frac{i \Delta t}{2} v_{ij} u_{ij}^{n+1} \\
    &= u_{ij}^n 
    + r \left[ u_{i+1,j}^{n}- 2 u_{ij}^{n} + u_{i-1,j}^{n} \right] \\
    &+ r \left[ u_{i,j+1}^{n}- 2 u_{ij}^{n} + u_{i,j-1}^{n} \right]
    - \frac{i \Delta t}{2} v_{ij} u_{ij}^{n}.
\end{align}
Here $r \equiv \frac{i \Delta t}{2h^2}$. $i$ indexes are not to be confused with the imaginary unit $i$. A more comprehensive analytical derivation can be found in appendix \ref{appendix:analytic}. Considering the case with our specific boundary conditions, this can be expressed as the matrix equation
\begin{equation}
    A \vec{u}^{\text{ }n+1} = B \vec{u}^{\text{ }n}.
\end{equation}
% ===========================================
\section{Results}\label{sec:results}
%
// prob 7 plots
\begin{figure}[H]
    \centering
    \includegraphics[width=.5\textwidth]{../figures/}
    \caption{Deviation of the total probability from 1.0 as a function of time. Simulation settings: $h = 0.005, \Delta t = 2.5 \times 10^{-5}, T = 0.008, x_c = 0.25, \sigma_x = 0.05, p_x = 200, y_c = 0.5, \sigma_y = 0.05, p_y = 0, v_0 = 0$.}
    \label{fig:}
\end{figure}

\begin{figure}[H]
    \centering
    \includegraphics[width=.5\textwidth]{../figures/}
    \caption{Deviation of the total probability from 1.0 as a function of time. Simulation settings: $h = 0.005, \Delta t = 2.5 \times 10^{-5}, T = 0.008, x_c = 0.25, \sigma_x = 0.05, p_x = 200, y_c = 0.5, \sigma_y = 0.10, p_y = 0, v_0 = 1 \times 10^{10}$.}
    \label{fig:}
\end{figure}

//prob 8 color map plots
\begin{figure}[H]
    \centering
    \includegraphics[width=.5\textwidth]{../figures/}
    \caption{Color map of the 2D probability distribution Simulation settings: $h = 0.005, \Delta t = 2.5 \times 10^{-5}, T = 0.008, x_c = 0.25, \sigma_x = 0.05, p_x = 200, y_c = 0.5, \sigma_y = 0.10, p_y = 0, v_0 = 1 \times 10^{10}$.}
    \label{fig:}
\end{figure}



% ===========================================
\section{Discussion}\label{sec:discussion}
%


% ===========================================
\section{Conclusion}\label{sec:conclusion}

\onecolumngrid

%\bibliographystyle{apalike}
\bibliography{ref}

\newpage
\appendix

\section{Analytical discretization of the 2+1 dimensional wave equation} \label{appendix:analytic}

The Schrödinger equation written as
\begin{equation}
    i \frac{\partial u}{\partial t} = - \frac{\partial^2 u}{\partial x^2} - \frac{\partial^2 u}{\partial y^2} + v(x,y)u, \label{eq:wave_eq_appendix}
\end{equation}
can be discretized using the Crank - Nicholson (CN) scheme. This involves using the forward Euler approximation for the time dimension and a linear combination of the second order center difference in the current and next time step.
\begin{equation}
    u(t + \Delta t ) = u(t) + \Delta t \frac{\partial u}{\partial t} + O(\Delta t^2)
\end{equation}
results in the forward Euler approximation
\begin{equation}
    \frac{\partial u}{\partial t} = \frac{u_{ij}^{n+1} - u_{ij}^n}{\Delta t} + O(\Delta t). \label{eq:partial_t}
\end{equation}
Similarly, using the Taylor expansion for the second order spatial derivative results in
\begin{equation}
    \frac{\partial^2 u}{\partial x^2} = \frac{u_{i+1,j} -2u_{ij} + u_{i-1,j}}{h^2} + O(\Delta t^2).
\end{equation}

For the CN scheme we evaluate the spatial differences in both the current and next time step to then use linear combination (using ``The $\theta$ rule'' \cite{compendium} for $\theta = \frac{1}{2}$). This gives us
\begin{equation}
    \frac{\partial^2 u}{\partial x^2} = \frac{1}{2} \left[ \frac{u_{i+1,j}^{n+1} -2u_{ij}^{n+1} + u_{i-1,j}^{n+1}}{h^2} 
    + \frac{u_{i+1,j}^{n} -2u_{ij}^{n} + u_{i-1,j}^{n}}{h^2} \right] \label{eq:partial_x}
\end{equation}
for the $x$ dimension. Equivalently, for the $y$ dimension we have
\begin{equation}
    \frac{\partial^2 u}{\partial y^2} = \frac{u_{i,j+1} -2u_{ij} + u_{i,j-1}}{h^2}
\end{equation}
and evaluated in the current and next time step for CN
\begin{equation}
    \frac{\partial^2 u}{\partial y^2} = \frac{1}{2} \left[ \frac{u_{i,j+1}^{n+1} -2u_{ij}^{n+1} + u_{i,j-1}^{n+1}}{h^2}
    + \frac{u_{i,j+1}^{n} -2u_{ij}^{n} + u_{i,j-1}^{n}}{h^2} \right]. \label{eq:partial_y}
\end{equation}
Now to discretize eq. \ref{eq:wave_eq_appendix} we insert our results from eq. \ref{eq:partial_t}, \ref{eq:partial_x} and \ref{eq:partial_y}. This gives
\begin{align}
    i \frac{u_{ij}^{n+1} - u_{ij}^n}{\Delta t} 
    =& - \frac{1}{2} \left[ \frac{u_{i+1,j}^{n+1} -2u_{ij}^{n+1} + u_{i-1,j}^{n+1}}{h^2} 
    + \frac{u_{i+1,j}^{n} -2u_{ij}^{n} + u_{i-1,j}^{n}}{h^2} \right] \\
     -& \frac{1}{2} \left[ \frac{u_{i,j+1}^{n+1} -2u_{ij}^{n+1} + u_{i,j-1}^{n+1}}{h^2}
     + \frac{u_{i,j+1}^{n} -2u_{ij}^{n} + u_{i,j-1}^{n}}{h^2} \right]
     + \frac{1}{2} \left[ v_{ij}u_{ij}^{n+1} + v_{ij}u_{ij}^n \right], 
\end{align}
where the whole RHS is evaluated in the current and next time step including the last term. Then multiplying by $i\Delta t$ on both sides (remembering that $i^2 = -1$), we have
\begin{align}
    - u_{ij}^{n+1} + u_{ij}^n 
    =& - \frac{i\Delta t}{2} \left[ \frac{u_{i+1,j}^{n+1} -2u_{ij}^{n+1} + u_{i-1,j}^{n+1}}{h^2} 
    + \frac{u_{i+1,j}^{n} -2u_{ij}^{n} + u_{i-1,j}^{n}}{h^2} \right] \\
     -& \frac{i\Delta t}{2} \left[ \frac{u_{i,j+1}^{n+1} -2u_{ij}^{n+1} + u_{i,j-1}^{n+1}}{h^2}
     + \frac{u_{i,j+1}^{n} -2u_{ij}^{n} + u_{i,j-1}^{n}}{h^2} \right]
     + \frac{i\Delta t}{2} \left[ v_{ij}u_{ij}^{n+1} + v_{ij}u_{ij}^n \right]. 
\end{align}
Introducing  $r \equiv \frac{i \Delta t}{2h^2}$ we have
\begin{align}
    - u_{ij}^{n+1} + u_{ij}^n 
    =& - r \left[ {u_{i+1,j}^{n+1} -2u_{ij}^{n+1} + u_{i-1,j}^{n+1}} \right]
    - r \left[ {u_{i+1,j}^{n} -2u_{ij}^{n} + u_{i-1,j}^{n}}\right] \\
    -& r \left[ {u_{i,j+1}^{n+1} -2u_{ij}^{n+1} + u_{i,j-1}^{n+1}} \right]
    - r \left[ {u_{i,j+1}^{n} -2u_{ij}^{n} + u_{i,j-1}^{n}}\right]
    + \frac{i\Delta t}{2} v_{ij}u_{ij}^{n+1} + \frac{i\Delta t}{2} v_{ij}u_{ij}^n. 
\end{align}
Collecting all the $n+1$ terms on the LHS we have the final expression
\begin{align}
    u_{ij}^{n+1} - r \left[ u_{i+1,j}^{n+1}- 2 u_{ij}^{n+1} + u_{i-1,j}^{n+1} \right] 
    - r \left[ u_{i,j+1}^{n+1}- 2 u_{ij}^{n+1} + u_{i,j-1}^{n+1} \right] 
    + \frac{i \Delta t}{2} v_{ij} u_{ij}^{n+1} \\
    = u_{ij}^n 
    + r \left[ u_{i+1,j}^{n}- 2 u_{ij}^{n} + u_{i-1,j}^{n} \right] 
    + r \left[ u_{i,j+1}^{n}- 2 u_{ij}^{n} + u_{i,j-1}^{n} \right]
    - \frac{i \Delta t}{2} v_{ij} u_{ij}^{n},
\end{align}
where $r \equiv \frac{i \Delta t}{2h^2}$. 

\end{document}